%% Generated by Sphinx.
\def\sphinxdocclass{report}
\documentclass[letterpaper,10pt,english]{sphinxmanual}
\ifdefined\pdfpxdimen
   \let\sphinxpxdimen\pdfpxdimen\else\newdimen\sphinxpxdimen
\fi \sphinxpxdimen=.75bp\relax

\usepackage[utf8]{inputenc}
\ifdefined\DeclareUnicodeCharacter
 \ifdefined\DeclareUnicodeCharacterAsOptional
  \DeclareUnicodeCharacter{"00A0}{\nobreakspace}
  \DeclareUnicodeCharacter{"2500}{\sphinxunichar{2500}}
  \DeclareUnicodeCharacter{"2502}{\sphinxunichar{2502}}
  \DeclareUnicodeCharacter{"2514}{\sphinxunichar{2514}}
  \DeclareUnicodeCharacter{"251C}{\sphinxunichar{251C}}
  \DeclareUnicodeCharacter{"2572}{\textbackslash}
 \else
  \DeclareUnicodeCharacter{00A0}{\nobreakspace}
  \DeclareUnicodeCharacter{2500}{\sphinxunichar{2500}}
  \DeclareUnicodeCharacter{2502}{\sphinxunichar{2502}}
  \DeclareUnicodeCharacter{2514}{\sphinxunichar{2514}}
  \DeclareUnicodeCharacter{251C}{\sphinxunichar{251C}}
  \DeclareUnicodeCharacter{2572}{\textbackslash}
 \fi
\fi
\usepackage{cmap}
\usepackage[T1]{fontenc}
\usepackage{amsmath,amssymb,amstext}
\usepackage{babel}
\usepackage{times}
\usepackage[Bjarne]{fncychap}
\usepackage[dontkeepoldnames]{sphinx}

\usepackage{geometry}

% Include hyperref last.
\usepackage{hyperref}
% Fix anchor placement for figures with captions.
\usepackage{hypcap}% it must be loaded after hyperref.
% Set up styles of URL: it should be placed after hyperref.
\urlstyle{same}

\addto\captionsenglish{\renewcommand{\figurename}{Fig.}}
\addto\captionsenglish{\renewcommand{\tablename}{Table}}
\addto\captionsenglish{\renewcommand{\literalblockname}{Listing}}

\addto\captionsenglish{\renewcommand{\literalblockcontinuedname}{continued from previous page}}
\addto\captionsenglish{\renewcommand{\literalblockcontinuesname}{continues on next page}}

\addto\extrasenglish{\def\pageautorefname{page}}

\setcounter{tocdepth}{1}



\title{test Documentation}
\date{Feb 12, 2018}
\release{}
\author{iottester}
\newcommand{\sphinxlogo}{\vbox{}}
\renewcommand{\releasename}{Release}
\makeindex

\begin{document}

\maketitle
\sphinxtableofcontents
\phantomsection\label{\detokenize{index::doc}}


Contents:


\chapter{Welcome!}
\label{\detokenize{intro:welcome}}\label{\detokenize{intro::doc}}\label{\detokenize{intro:welcome-to-ioraptor-documentation}}
Welcome to the ioRaptor Documentation. Please select one of the topics.


\chapter{Text Formating}
\label{\detokenize{textformating::doc}}\label{\detokenize{textformating:text-formating}}

\section{Bold, Italics}
\label{\detokenize{textformating:bold-italics}}
Here we start with basics:

I like it when it is \sphinxstyleemphasis{italics}

I see it when it is \sphinxstylestrong{boldface}


\section{Code Blocks}
\label{\detokenize{textformating:code-blocks}}
Hardcore coding is easy with \sphinxcode{Serial.println("Hello World")}


\section{Lists}
\label{\detokenize{textformating:lists}}\begin{itemize}
\item {} 
This is a bulleted list.

\item {} 
It has two items, the second item uses two lines.

\end{itemize}
\begin{enumerate}
\item {} 
This is a numbered list.

\item {} 
It has two items too.

\end{enumerate}

Another list:
\begin{enumerate}
\item {} 
This is a numbered list.

\item {} 
It has two items too.

\end{enumerate}

Nested List:
\begin{itemize}
\item {} 
This list

\item {} 
includes

\item {} 
a nested list

\item {} 
and some subitems

\item {} 
and here the parent list continues

\end{itemize}


\section{Terms and Definitions}
\label{\detokenize{textformating:terms-and-definitions}}
term (up to a line of text)
Definition of the term, which must be indented

and can even consist of multiple paragraphs

next term
Description.

\begin{DUlineblock}{0em}
\item[] These lines are
\item[] broken exactly like in
\item[] the source file.
\item[] 
\end{DUlineblock}

This is a normal text paragraph. The next paragraph is a code sample:

\fvset{hllines={, ,}}%
\begin{sphinxVerbatim}[commandchars=\\\{\}]
\PYG{n}{It} \PYG{o+ow}{is} \PYG{o+ow}{not} \PYG{n}{processed} \PYG{o+ow}{in} \PYG{n+nb}{any} \PYG{n}{way}\PYG{p}{,} \PYG{k}{except}
\PYG{n}{that} \PYG{n}{the} \PYG{n}{indentation} \PYG{o+ow}{is} \PYG{n}{removed}\PYG{o}{.}
\end{sphinxVerbatim}

It can span multiple lines.

This is a normal text paragraph again.


\section{Tables}
\label{\detokenize{textformating:tables}}

\begin{savenotes}\sphinxattablestart
\centering
\begin{tabulary}{\linewidth}[t]{|T|T|T|T|}
\hline
\sphinxstylethead{\sphinxstyletheadfamily 
Header row, column 1
\unskip}\relax &\sphinxstylethead{\sphinxstyletheadfamily 
Header 2
\unskip}\relax &\sphinxstylethead{\sphinxstyletheadfamily 
Header 3
\unskip}\relax &\sphinxstylethead{\sphinxstyletheadfamily 
Header 4
\unskip}\relax \\
\hline
body row 1, column 1
&
column 2
&
column 3
&
column 4
\\
\hline
body row 2
&
…
&
…
&\\
\hline
\end{tabulary}
\par
\sphinxattableend\end{savenotes}


\begin{savenotes}\sphinxattablestart
\centering
\begin{tabulary}{\linewidth}[t]{|T|T|T|}
\hline
\sphinxstylethead{\sphinxstyletheadfamily 
A
\unskip}\relax &\sphinxstylethead{\sphinxstyletheadfamily 
B
\unskip}\relax &\sphinxstylethead{\sphinxstyletheadfamily 
A and B
\unskip}\relax \\
\hline
False
&
False
&
False
\\
\hline
True
&
False
&
False
\\
\hline
False
&
True
&
False
\\
\hline
True
&
True
&
True
\\
\hline
\end{tabulary}
\par
\sphinxattableend\end{savenotes}


\section{Links and Photos}
\label{\detokenize{textformating:links-and-photos}}
This is a paragraph that contains \sphinxhref{http://thingforward.io/}{a link}.

\noindent\sphinxincludegraphics{{ioraptor}.png}


\section{Footnotes and Comments}
\label{\detokenize{textformating:footnotes-and-comments}}
Footnotes:

Lorem ipsum %
\begin{footnote}[1]\sphinxAtStartFootnote
Text of the first footnote.
%
\end{footnote} dolor sit amet … %
\begin{footnote}[2]\sphinxAtStartFootnote
Text of the second footnote.
%
\end{footnote}

Comment:


\chapter{Topic 3}
\label{\detokenize{topic3:topic-3}}\label{\detokenize{topic3::doc}}
Empty Topic! :(


\chapter{Topic 4}
\label{\detokenize{topic4::doc}}\label{\detokenize{topic4:topic-4}}
Empty Topic! :(


\chapter{Topic 6}
\label{\detokenize{topic5::doc}}\label{\detokenize{topic5:topic-6}}
Empty Topic! :(


\chapter{Topic 3}
\label{\detokenize{topic6:topic-3}}\label{\detokenize{topic6::doc}}
Empty Topic! :(


\chapter{Contents}
\label{\detokenize{index:contents}}\begin{itemize}
\item {} 
\DUrole{xref,std,std-ref}{genindex}

\item {} 
\DUrole{xref,std,std-ref}{modindex}

\item {} 
\DUrole{xref,std,std-ref}{search}

\end{itemize}



\renewcommand{\indexname}{Index}
\printindex
\end{document}